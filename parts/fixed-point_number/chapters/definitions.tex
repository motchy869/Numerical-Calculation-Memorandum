\chapter{定義}
    \newcommand{\Ni}{{N_\text{int}}}
    \newcommand{\Nd}{{N_\text{dec}}}
    \section{前提}
        \label{fxd-pt_num::def::prem}
        \begin{enumerate}
            \item 「2 の補数表現」など、大学の理工系の教養課程で扱われる内容は前提とし、本書では一々定義しない。
            \item 基数は 2 である。
        \end{enumerate}
    \section{構造}
        固定小数点数とは、1 ビットの記憶領域 bit-cell の有限列と、その各領域に格納される数の組である(乱暴に言えば、容器と中身の両方を合わせたものである)。
        bit-cell の列は「整数部」と「小数部」という連続した部分配列に二分される。
        整数部は必ず存在する。小数部は存在しなくてもよい。
        整数部と小数部の bit-cell の個数(以降は「ビット数」と呼ぶ)をそれぞれ $\Ni(\geq 1), \Nd(\geq 0)$ とする。
        整数部と小数部の bit-cell の列は添え字が付けられる。
        整数部の添え字は $0,1,\dots,\Ni$ である。
        小数部が存在するとき、その添え字は $-1,-2,\dots,-\Nd$ である。
        整数部の最上位ビットを「符号ビット」と呼ぶ。
        これの意味は後述される。
        整数部のビット数が $\Ni$、小数部のビット数が $\Nd$ である形式を「 $\Ni$Q$\Nd$ 形式」と呼ぶ。\footnote{ARM variant の Q format はこの形式を採用している。}
        \par
        固定小数点数 $a$ と整数 $k$ について、 $k$ が $a$ の添え字の範囲に含まれるとき、$a[k]$ は第 $k$ bit-cell に格納されている数 (0 or 1) を表す。
        $k$ が $a$ の添え字の範囲外であるとき、 $a[k]$ は 0 であると定義する。
    \section{対応する有理数}
        $a$ を固定小数点数とする。
        符号ビットが 0 であるとき、「 $a$ に対応する有理数 (a rational number corresponding to $a$)」は次式で定義される。
        \[ \sum_{k=-\Nd}^{\Ni-1} a[k]2^k = \parens*{\sum_{k=0}^{\Ni+\Nd-1} a[k]2^k}/2^\Nd \]
        符号ビットが 1 であるとき、「 $a$ に対応する有理数」は次式で定義される。
        \[ -\parens*{2^\Ni - \sum_{k=-\Nd}^{\Ni-1} a[k]2^k} = -\parens*{2^{\Ni+\Nd} - \sum_{k=0}^{\Ni+\Nd-1} a[k]2^k}/2^\Nd \]
        \par
        以上の変換規則は、$a$ の整数部と小数部を連結して 2 の補数表現の符号付き整数と見做したものに対応する整数を $2^\Nd$ で割ったものと等しい。
    \section{特性に関する写像}
        固定小数点数 $a$ の整数部と小数部のビット数がそれぞれ $\Ni, \Nd$ であるとする。
        次の写像を定義する。
        \begin{enumerate}
            \item 整数部のビット幅:$\IntPartBW{a} \coloneqq \Ni$
            \item 小数部のビット幅:$\DecPartBW{a} \coloneqq \Nd$
            \item 全体のビット幅:$\TotalBW{a} \coloneqq \Ni + \Nd$
            \item 対応する有理数:$\Rat{a} \coloneqq \text{a rational number corresponding to }a$
        \end{enumerate}
        次に、有理数 $r$ から固定小数点数への写像に関わるいくつかの写像を定義する。
        尚、有理数から固定小数点数への写像の手続きは \ref{fxd-pt_num::def::prem} で述べたように既知であるとして説明を割愛する。
        $r$ を表現するのに必要十分な整数部のビット数は $\ceil{\log_2(\abs{\floor{r}}+1)}+1$ である(最後の +1 は符号ビットのため)\footnote{$r$ が負、例えば $-1.5$ のときは $-2+0.5$ のように「整数 + 非負の小数」に分けると理解し易い。}。
        これを以て「整数部の必要ビット数」と定義する。
        また、小数部のビットを添え字 -1 から下に向かって見てゆくとき、次の 2 通りがあり得る:
        \begin{enumerate}
            \item ある添え字 $k$ 以下では常にビットが 0 となる。
            \item どのような添え字 $k$ をとっても、それより小さい添え字で 1 となるビットが存在する(所謂、無限小数)。
        \end{enumerate}
        1 の場合、$-k-1$ を以て「小数部の必要ビット数」と定義する。
        2 の場合、「小数部の必要ビット数」は無限大であると定義する。
    \section{積}
        \newcommand{\Nia}{{N_{\text{int},a}}}
        \newcommand{\Nda}{{N_{\text{dec},a}}}
        \newcommand{\Nib}{{N_{\text{int},b}}}
        \newcommand{\Ndb}{{N_{\text{dec},b}}}
        \subsection{定義}
            固定小数点数 $a, b$ について $a$ の整数部と小数部のビット数がそれぞれ $\Nia, \Nda$ であり、$b$ の整数部と小数部のビット数がそれぞれ $\Nib, \Ndb$ であるとする。
            $a$ と $b$ の積 $a\times b$ を「$\Rat{a}\Rat{b}$ を、整数部が $\Nia+\Nib$ ビット、小数部が $\Nda+\Ndb$ ビットの固定小数点数に写像して得られる数」として定義する。
            \par
        \subsection{性質}
            \newcommand{\Na}{{N_\text{a}}}
            \newcommand{\Nb}{{N_\text{b}}}
            前記の定義に従うと、常に $\Rat{a\times b} = \Rat{a}\Rat{b}$ が成り立つ。
            \begin{proof}
                \quad\par
                $a\times b$ の整数部のビット数 $\Nia+\Nib$ が $\Rat{a}\Rat{b}$ の整数部の必要ビット数以上であり、かつ $a\times b$ の小数部のビット数 $\Nda+\Ndb$ が $\Rat{a}\Rat{b}$ の小数部の必要ビット数以上であることを示せばよい。
                まず次式が成り立つ。
                \[ \Rat{a}\Rat{b} = \underbrace{2^\Nda\Rat{a}}_{\eqqcolon\alpha}\underbrace{2^\Ndb\Rat{b}}_{\eqqcolon\beta}/2^{\Nda+\Ndb} \]
                $\Na \coloneqq \Nia + \Nda,\; \Nb \coloneqq \Nib + \Ndb$ とすると次式が成り立つ。
                \begin{gather*}
                    -2^{\Na-1} \leq \alpha \leq 2^{\Na-1}-1,\quad -2^{\Nb-1} \leq \beta \leq 2^{\Nb-1}-1 \\
                    \therefore -2^{\Na+\Nb-2} + \min \braces*{2^{\Na-1},\; 2^{\Nb-1}} \leq \alpha\beta \leq 2^{\Na+\Nb-2} \\
                    \therefore -2^{\Nia+\Nib-2} + \min \braces*{2^{\Nia-\Ndb-1},\; 2^{\Nib-\Nda-1}} \leq \alpha\beta/2^{\Nda+\Ndb} \leq \Rat{a}\Rat{b} \leq 2^{\Nia+\Nib-2}
                \end{gather*}
                符号ビットを考慮して $\Rat{a}\Rat{b}$ の整数部の必要ビット数は $\Nia+\Nib$ 以下である。
                また、 $\alpha\beta$ は整数であるから $\Rat{a}\Rat{b}$ の小数部は $\alpha\beta$ を $2^{\Nda+\Ndb}$ で割ったものの小数部と等しい。
                よって $\Rat{a}\Rat{b}$ の小数部の必要ビット数は $\Nda+\Ndb$ 以下である。
            \end{proof}
