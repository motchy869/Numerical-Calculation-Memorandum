\chapter{Cholesky分解}
    \section{rank-one update}
        \begin{itembox}[l]{主張}
            $n\in\naturalNumbers,\;A\in\complexNumbers^{n\times n},\;A\succeq O,\;\bm{x}\in\complexNumbers^n$とし、$A$はHermite行列であるとする。
            $A+\bm{x}\bm{x}^*$に対してCholesky分解のアルゴリズムを適用すると$O(n^3)$の計算量を要する。
            しかし、$A$のCholesky分解$LL^*$が既に得られているとき、$A+\bm{x}\bm{x}^*$のCholesky分解を$O(n^2)$で得ることができる。
            $\bm{x}\bm{x}^*$の階数が1以下である(特に0となるのは$\bm{x}=\bm{0}$の時かつその時に限る)ことから、この方法は ``rank-one update'' と呼ばれている。
        \end{itembox}
        \begin{derivation*}
            方針としては、$n\times n$行列の rank-one update を$(n-1)\times(n-1)$行列の問題に帰着させ、以降同様に逐次的に行列の次数を縮小しながら解を構築する。
            このアルゴリズムの総計算量が$O(n^2)$となるのは明らかであろう。
            \par
            $A+\bm{x}\bm{x}^*$のCholesky分解を$FF^*$とする。
            $L$の第$i$列ベクトルを$\bm{l}_i = [0,\dots,0,l_{i,i},\dots,l_{n,i}]^\top\in\complexNumbers^{n\times n}$とし、同様に$F$の第$i$列ベクトルを$\bm{f}_i = [0,\dots,0,f_{i,i},\dots,f_{n,i}]^\top\in\complexNumbers^{n\times n}$とすると次式が成り立つ。
            \begin{align*}
                \sum_{i=1}^n \bm{f}_i\bm{f}_i^* &= \bm{x}\bm{x}^* + \sum_{i=1}^n \bm{l}_i\bm{l}_i^* \\
                \bm{f}_1\bm{f}_1^* + \sum_{i=2}^n \bm{f}_i\bm{f}_i^* &= \bm{x}\bm{x}^* + \bm{l}_1\bm{l}_1^* + \sum_{i=2}^n \bm{l}_i\bm{l}_i^* \tag{1}
            \end{align*}
            $\bm{f}_i\bm{f}_i^*,\;\bm{l}_i\bm{l}_i^*\;(i=2,3,\dots,n)$の第1行および第1列は0であるから、$\bm{f}_1\bm{f}_1^*$と$\bm{x}\bm{x}^* + \bm{l}_1\bm{l}_1^*$の第1行および第1列が一致する。
            これより次式が成り立つ。
            \[ f_{1,1} = \sqrt{l_{1,1}^2 + |x_1|^2} \eqqcolon r,\; f_{k,1} = \frac{1}{r}\left(l_{1,1}l_{k,1} + \overline{x_1}x_k\right) \; (k=2,3,\dots,n) \tag{2} \]
            ただし$L$の対角成分が非負の実数であることを前提としている。
            以上より、$\tilde{\bm{l}_1} \coloneqq [0,l_{2,1},l_{3,1},\dots,l_{n,1}]^\top,\;\tilde{\bm{x}} \coloneqq [0,x_2,x_3,\dots,x_n]^\top$とすると次式が成り立つ。
            \[ \bm{f}_1 = r\bm{e}_1 + \frac{l_{1,1}}{r}\tilde{\bm{l}_1} + \frac{\conj{x_1}}{r}\tilde{\bm{x}} \]
            ここに$\bm{e}_1$は第1要素が1で他は0であるベクトルである。
            $\bm{f}_1\bm{f}_1^*$の右下$(n-1)\times(n-1)$行列を評価すると次式を得る。
            \begin{align*}
                &\phantom{=} \frac{1}{r^2}\left(l_{1,1}\tilde{\bm{l}_1} + \conj{x_1}\tilde{\bm{x}}\right) = \frac{1}{r^2} \left(l_{1,1}^2\tilde{\bm{l}_1}\tilde{\bm{l}_1}^* + l_{1,1}x_1\tilde{\bm{l}_1}\tilde{\bm{x}}^* + \abs{x_1}^2\tilde{\bm{x}}\tilde{\bm{x}}^* + l_{1,1}\overline{x_1}\tilde{\bm{x}}\tilde{\bm{l}_1}^*\right) \\
                &= \left(1 - \frac{\abs{x_1}^2}{r^2}\right)\tilde{\bm{l}_1}\tilde{\bm{l}_1}^* + \frac{l_{1,1}x_1}{r^2}\tilde{\bm{l}_1}\tilde{\bm{x}}^* + \left(1 - \frac{l_{1,1}^2}{r^2}\right)\tilde{\bm{x}}\tilde{\bm{x}}^* + \frac{\overline{x_1}l_{1,1}}{r^2}\tilde{\bm{x}}\tilde{\bm{l}_1}^* \\
                &= \tilde{\bm{l}_1}\tilde{\bm{l}_1}^* + \tilde{\bm{x}}\tilde{\bm{x}}^* - \frac{1}{r^2}\left(\abs{x_1}^2\tilde{\bm{l}_1}\tilde{\bm{l}_1}^* + l_{1,1}^2\tilde{\bm{x}}\tilde{\bm{x}}^* - x_1l_{1,1}\tilde{\bm{l}_1}\tilde{\bm{x}}^* - \overline{x_1}l_{1,1}\tilde{\bm{x}}\tilde{\bm{l}_1}^*\right) \\
                &= \tilde{\bm{l}_1}\tilde{\bm{l}_1}^* + \tilde{\bm{x}}\tilde{\bm{x}}^* - \bm{y}\bm{y}^* \quad \text{where} \quad \bm{y} = \frac{1}{r}\left(l_{1,1}\tilde{\bm{x}} - x_1\tilde{\bm{l}_1}\right)
            \end{align*}
            上式の$\tilde{\bm{l}_1}\tilde{\bm{l}_1}^* + \tilde{\bm{x}}\tilde{\bm{x}}^*$は$\bm{x}\bm{x}^* + \bm{l}_1\bm{l}_1^*$の右下$(n-1)\times(n-1)$行列である。
            以上より次式が成り立つ。
            \[ \bm{f}_1\bm{f}_1^* = \bm{x}\bm{x}^* + \bm{l}_1\bm{l}_1^* - \bm{y}\bm{y}^* \]
            これを式(1)に適用して次式を得る。
            \[ \sum_{i=2}^n \bm{f}_i\bm{f}_i^* = \bm{y}\bm{y}^* + \sum_{i=2}^n \bm{l}_i\bm{l}_i^* \]
            これは$(n-1)\times(n-1)$行列の rank-one update である。
            このようにして行列の次数を逐次的に縮小し、最後はスカラーの計算に帰着する。
            次数$k$の問題に対し式(2)の計算量は$O(k)$であるから、このアルゴリズムの総計算量は$n(n+1)/2$に比例する。
            \qed
        \end{derivation*}
        このアルゴリズムのJulia 1.8.0による実装例を\inlineCode{Cholesky-decomposition_rank-one_update.ipynb}に記した。
        本文書のGitリポジトリ内でファイル検索すれば見つかる。
