\chapter{LDL分解}
    \section{rank-one update}
        \begin{itembox}[l]{主張}
            $n\in\naturalNumbers,\;A\in\complexNumbers^{n\times n},\;A\succeq O,\;\bm{x}\in\complexNumbers^n$とし、$A$はHermite行列であるとする。
            $A+\bm{x}\bm{x}^*$に対してLDL分解のアルゴリズムを適用すると$O(n^3)$の計算量を要する。
            しかし、$A$のLDL分解$LDL^*$が既に得られているとき、$A+\bm{x}\bm{x}^*$のLDL分解を$O(n^2)$で得ることができる。
            $\bm{x}\bm{x}^*$の階数が1以下である(特に0となるのは$\bm{x}=\bm{0}$の時かつその時に限る)ことから、この方法は ``rank-one update'' と呼ばれている。
        \end{itembox}
        \begin{derivation*}
            方針はCholesky分解の rank-one update と同様である。
            $A+\bm{x}\bm{x}^*$のLDL分解を$FGF^*$とする。
            $D,G$の第$i$対角成分をそれぞれ$d_i,g_i$とする。
            但し$d_i\geq 0$を前提とする。
            $L$の第$i$列ベクトルを$\bm{l}_i = [0,\dots,0,1,l_{i+1,i},\dots,l_{n,i}]^\top\in\complexNumbers^{n\times n}$とし、同様に$F$の第$i$列ベクトルを$\bm{f}_i = [0,\dots,0,1,f_{i+1,i},\dots,f_{n,i}]^\top\in\complexNumbers^{n\times n}$とすると次式が成り立つ。
            \begin{align*}
                \sum_{i=1}^n \bm{f}_i g_i\bm{f}_i^* &= \bm{x}\bm{x}^* + \sum_{i=1}^n \bm{l}_i d_i\bm{l}_i^* \\
                \bm{f}_1 g_1\bm{f}_1^* + \sum_{i=2}^n \bm{f}_i g_i\bm{f}_i^* &= \bm{x}\bm{x}^* + \bm{l}_1 d_1\bm{l}_1^* + \sum_{i=2}^n \bm{l}_i d_i\bm{l}_i^* \tag{1}
            \end{align*}
            $\bm{f}_i g_i\bm{f}_i^*,\;\bm{l}_i d_i\bm{l}_i^*\;(i=2,3,\dots,n)$の第1行および第1列は0であるから、$\bm{f}_1 g_1\bm{f}_1^*$と$\bm{x}\bm{x}^* + \bm{l}_1 d_1\bm{l}_1^*$の第1行および第1列が一致する。
            これより次式が成り立つ。
            \[ g_1 = d_1 + \abs{x_1}^2 \eqqcolon g,\; f_{k,1} = \frac{1}{g}\left(d_1 l_{k,1} + \overline{x_1}x_k\right) \; (k=2,3,\dots,n) \tag{2} \]
            以上より、$\tilde{\bm{l}_1} \coloneqq [0,l_{2,1},l_{3,1},\dots,l_{n,1}]^\top,\;\tilde{\bm{x}} \coloneqq [0,x_2,x_3,\dots,x_n]^\top$とすると次式が成り立つ。
            \[ \bm{f}_1 = \bm{e}_1 + \frac{d_1}{g}\tilde{\bm{l}_1} + \frac{\conj{x_1}}{g}\tilde{\bm{x}} \]
            ここに$\bm{e}_1$は第1要素が1で他は0であるベクトルである。
            $\bm{f}_1 g_1\bm{f}_1^*$の右下$(n-1)\times(n-1)$行列を評価すると次式を得る。
            \begin{align*}
                &\phantom{=} \frac{1}{g}\left(d_1\tilde{\bm{l}_1} + \tilde{\bm{x}}\tilde{\bm{x}}^*\right)\left(d_1\tilde{\bm{l}_1} + \tilde{\bm{x}}\tilde{\bm{x}}^*\right)^* = \frac{d_1}{g}\tilde{\bm{l}_1}d_1\tilde{\bm{l}_1}^* + \frac{\abs{x_1}^2}{g}\tilde{\bm{x}}\tilde{\bm{x}}^* + \frac{d_1}{g}\left(x_1\tilde{\bm{l}_1}\tilde{\bm{x}}^* + \conj{x_1}\tilde{\bm{x}}\tilde{\bm{l}_1}^*\right) \\
                &= \frac{g - \abs{x_1}^2}{g}\tilde{\bm{l}_1}d_1\tilde{\bm{l}_1}^* + \frac{g-d_1}{g}\tilde{\bm{x}}\tilde{\bm{x}}^* + \frac{d_1}{g}\left(x_1\tilde{\bm{l}_1}\tilde{\bm{x}}^* + \conj{x_1}\tilde{\bm{x}}\tilde{\bm{l}_1}^*\right) \\
                &= \tilde{\bm{l}_1}d_1\tilde{\bm{l}_1}^* + \tilde{\bm{x}}\tilde{\bm{x}}^* - \frac{d_1}{g}\left[\abs{x_1}^2\tilde{\bm{l}_1}\tilde{\bm{l}_1}^* + \tilde{\bm{x}}\tilde{\bm{x}}^* - x_1\tilde{\bm{l}_1}\tilde{\bm{x}}^* - \conj{x_1}\tilde{\bm{x}}\tilde{\bm{l}_1}^*\right] \\
                &= \tilde{\bm{l}_1}d_1\tilde{\bm{l}_1}^* + \tilde{\bm{x}}\tilde{\bm{x}}^* - \bm{y}\frac{d_1}{g}\bm{y}^* \quad \text{where} \quad \bm{y} = x_1\tilde{\bm{l}_1} - \tilde{\bm{x}}
            \end{align*}
            上式の$\tilde{\bm{l}_1}d_1\tilde{\bm{l}_1}^* + \tilde{\bm{x}}\tilde{\bm{x}}^*$は$\bm{x}\bm{x}^* + \bm{l}_1 d_1\bm{l}_1^*$の右下$(n-1)\times(n-1)$行列である。
            以上より次式が成り立つ。
            \[ \bm{f}_1 g_1\bm{f}_1^* = \bm{x}\bm{x}^* + \bm{l}_1 d_1\bm{l}_1^* - \bm{y}\frac{d_1}{g}\bm{y}^* \]
            これを式(1)に適用して次式を得る。
            \[ \sum_{i=2}^n \bm{f}_i g_i\bm{f}_i^* = \bm{y}\frac{d_1}{g}\bm{y}^* + \sum_{i=2}^n \bm{l}_i d_i\bm{l}_i^* \]
            これは$(n-1)\times(n-1)$行列の rank-one update である。
            このようにして行列の次数を逐次的に縮小し、最後はスカラーの計算に帰着する。
            次数$k$の問題に対し式(2)の計算量は$O(k)$であるから、このアルゴリズムの総計算量は$n(n+1)/2$に比例する。
            \qed
        \end{derivation*}
        このアルゴリズムのJulia 1.8.0による実装例を\inlineCode{LDL-decomposition_rank-one_update.ipynb}に記した。
        本文書のGitリポジトリ内でファイル検索すれば見つかる。
