\chapter{Cholesky分解}
    \section{rank-one update}
        \begin{itembox}[l]{主張}
            $n\in\naturalNumbers,\;A\in\complexNumbers^{n\times n},\;A\succeq O,\;\bm{x}\in\complexNumbers^n$とし、$A$はHermite行列であるとする。
            $A+\bm{x}\bm{x}^*$に対してCholesky分解のアルゴリズムを適用すると$O(n^3)$の計算量を要する。
            しかし、$A$のCholesky分解$LL^*$が既に得られているとき、$A+\bm{x}\bm{x}^*$のCholesky分解を$O(n^2)$で得ることができる。
            $\bm{x}\bm{x}^*$の階数が1以下である(特に0となるのは$\bm{x}=\bm{0}$の時かつその時に限る)ことから、この方法は ``rank-one update'' と呼ばれている。
        \end{itembox}
        \begin{derivation*}
            方針としては、$n\times n$行列の rank-one update を$(n-1)\times(n-1)$行列の問題に帰着させ、再帰的に解を構築する。
            \par
            $A+\bm{x}\bm{x}^*$のCholesky分解を$FF^*$とする。
            $L$の第$i$列ベクトルを$\bm{l}_i = [l_{1,1},l_{2,1},\dots,l_{n,1}]^\top$とし、同様に$F$の第$i$列ベクトルを$\bm{f}_i = [f_{1,1},f_{2,1},\dots,f_{n,1}]^\top$とすると次式が成り立つ。
            \begin{align*}
                \sum_{i=1}^n \bm{f}_i\bm{f}_i^* &= \bm{x}\bm{x}^* + \sum_{i=1}^n \bm{l}_i\bm{l}_i^* \\
                \bm{f}_1\bm{f}_1^* + \sum_{i=2}^n \bm{f}_i\bm{f}_i^* &= \bm{x}\bm{x}^* + \bm{l}_1\bm{l}_1^* + \sum_{i=2}^n \bm{l}_i\bm{l}_i^*
            \end{align*}
            $\bm{f}_i\bm{f}_i^*,\;\bm{l}_i\bm{l}_i^*\;(i=2,3,\dots,n)$の第1行および第1列は0であるから、$\bm{f}_1\bm{f}_1^*$と$\bm{x}\bm{x}^* + \bm{l}_1\bm{l}_1^*$の第1行および第1列が一致する。
            これより次式が成り立つ。
            \[ f_{1,1} = \sqrt{l_{1,1}^2 + |x_1|^2} \]
            \qed
        \end{derivation*}
