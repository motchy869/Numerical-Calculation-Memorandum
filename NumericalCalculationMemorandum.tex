\RequirePackage{plautopatch}
\RequirePackage[l2tabu, orthodox]{nag} % 古いコマンド, パッケージに対する警告

\documentclass[titlepage]{ltjsreport}
\usepackage{amsmath, amssymb, amsfonts, amsthm}
\usepackage{ascmac}
\usepackage{bbm}
\usepackage{bigdelim}
\usepackage{cases}
\usepackage{chngcntr}
\usepackage[usenames]{color}
\usepackage{comment}
\usepackage{currfile}
\usepackage{empheq}
\usepackage{fancybox}
\usepackage{graphicx}
\usepackage{here}
\usepackage[unicode,bookmarksopenlevel=4,bookmarksnumbered=true,colorlinks=true,linkcolor=blue,urlcolor=blue]{hyperref}
\usepackage{cleveref} % must be included after hyperref
\usepackage{ifthen}
\usepackage{mathtools}
\usepackage{multirow}
\usepackage{nameref}
\usepackage{siunitx}
\usepackage{tabularx, arydshln}
\usepackage{udline}
\usepackage{url}
\usepackage{zref-xr}

\theoremstyle{definition} % 定理環境:斜体なし
\newtheorem*{commentary}{解説}
\newtheorem*{attention}{注意}
\newtheorem*{solution}{解法}
\newtheorem*{definition}{定義}
\newtheorem*{derivation*}{導出}

% section, subsection, subsubsection の先頭に部番号が付くようにする
\counterwithin{chapter}{part}\renewcommand{\thechapter}{\thepart.\arabic{chapter}}
\counterwithin{section}{chapter}\renewcommand{\thesection}{\thechapter.\arabic{section}}
\counterwithin{subsection}{section}\renewcommand{\thesubsection}{\thesection.\arabic{subsection}}
\counterwithin{subsubsection}{subsection}\renewcommand{\thesubsubsection}{\thesubsection.\arabic{subsubsection}}

\renewcommand{\theequation}{\arabic{equation}} % jsbook によって書き換えられた \theequation を元に戻す

% section 毎に数式, 図番号をリセットする。番号表示は「part.chapter.section.section 内での通し番号」となる。
\counterwithin{equation}{section}
\counterwithin{figure}{section}

% 目次の番号とタイトルが重ならないようにする
% dottedtocline{A}{B}{C}のパラメータABC
%   A:目次を生成するレベル(\chapterはレベル0、\sectionはレベル1...)
%   B:一番外側からの左マージン
%   C:見出し番号が入るボックスの幅
\makeatletter
\renewcommand{\l@chapter}{\@dottedtocline{0}{0em}{7em}}
\renewcommand{\l@section}{\@dottedtocline{1}{1em}{7em}}
\renewcommand{\l@subsection}{\@dottedtocline{2}{2em}{6em}}
\renewcommand{\l@subsubsection}{\@dottedtocline{3}{3em}{5em}}
\makeatother

\setcounter{tocdepth}{4} % subsubsubsection まで目次に表示する

% cleveref settings
% see: [LaTeXの相互参照を賢くする (cleveref)](https://qiita.com/Hdan/items/8c59a7e0a3215ae32d74)
\crefname{equation}{式}{式} % {環境名}{単数形}{複数形}
\crefname{figure}{図}{図}
\crefname{table}{表}{表}
\crefname{section}{第}{第}
\creflabelformat{section}{#2#1節#3}
\crefname{subsection}{第}{第}
\creflabelformat{subsection}{#2#1小節#3}
\newcommand{\crefpairconjunction}{と}
\newcommand{\crefrangeconjunction}{から}
\newcommand{\crefmiddleconjunction}{,}
\newcommand{\creflastconjunction}{,および}

% document version number
\def\docVerMajor{0}
\def\docVerMinor{11}
\def\docVerPatch{1}
\def\docVerWip{} % \def\docVerWip{-wip}
\def\docVer{\docVerMajor.\docVerMinor.\docVerPatch\docVerWip}

\input{cmds/MtyLaTeXCmds/MtyLaTeXCmds}

\begin{document}
    \title{数値計算備忘録}
    \author{motchy}
    \date{\西暦 2022年8月23日 $\sim$ \today \\ver \docVer}
	\maketitle

    \tableofcontents

    \part{線形代数}
    \chapter{Gauss-Seidel法}
    \section{定義}
        以下に述べる定義はWikipediaの英語記事\href{https://en.wikipedia.org/wiki/Gauss%E2%80%93Seidel_method}{Gauss Seidel method}からの引用である。
        \par
        $n\in\naturalNumbers,\;A\in\complexNumbers^{n\times n},\;\bm{b}\in\complexNumbers^n$ とする。$A$ は正定値対称、または狭義優対角であるとする。Gauss-Seidel法とは、線型方程式 $A\bm{x}=\bm{b}$ の解を求める反復法である。$\bm{x}_1\in\complexNumbers^n$ を任意の初期解とし、次の漸化式で解候補を更新してゆく。
        \[ L_* \bm{x}_{k+1} = -U\bm{x}_k \quad (k=1,2,\dots)\]
        ここに $L_*$ は $A$ の対角成分およびその下側の要素からなる下三角行列であり、 $U$ は $A$ の対角成分の上側の要素からなる上三角行列である。
    \section{係数行列が狭義優対角ならば厳密解に収束すること}
        \begin{proof}
            \quad\par
            $A$ の次数を $n$ とする。$\mathring{\bm{x}}$ を厳密解とすると $L_* \mathring{\bm{x}} = \bm{b} - U\mathring{\bm{x}}$ である。
            これを解の更新式から減じると次式を得る。
            \[ L_* (\bm{x}_{k+1} - \mathring{\bm{x}}) = -U(\bm{x}_k - \mathring{\bm{x}}) \tag{1} \]
            $\bm{v}_k \coloneqq \bm{x}_k - \mathring{\bm{x}}$ とおくと、式 (1) より次式が成り立つ。
            \[ L_* \bm{v}_{k+1} = -U\bm{v}_k \tag{2} \]
            $M_k \coloneqq \max_{i=1,\dots,n} |v_{k,i}|\;(v_{k,i}$ は $\bm{v}_k$ の第 $i$ 要素)とする。
            次の2つが同時に成り立つことが、$\bm{v}_k$ が $\bm{0}_n$ に収束するための十分条件である。
            \begin{enumerate}
                \item ある $k \in \mathbb{N}$ に対して $M_k = 0$ ならば $M_l = 0\;(l=k+1,k+2,\dots)$
                \item 適当な $0 < \alpha < 1$ が存在して $M_k > 0 \Rightarrow M_{k+1} < \alpha M_k$
            \end{enumerate}
            $L_*$ が正則であることと式 (2) より直ちに 1. が成り立つ。
            次に 2. を数学的帰納法で示す。
            $\tilde{\alpha}$ を次式で定義する。
            \[ \tilde{\alpha} \coloneqq \min_{i=1,2,\dots,n} \frac{1}{|a_{i,i}|} \sum_{j=1,\dots,n \wedge j\neq i} |a_{i,j}| \]
            $A$ は優対角だから $0 < \tilde{\alpha} < 1$ である。
            \begin{align*}
                a_{1,1} v_{k+1, 1} &= -\sum_{j=2}^n a_{1,j}v_{k,j} \\
                |a_{1,1}| |v_{k+1, 1}| &= \left|\sum_{j=2}^n a_{1,j}v_{k,j}\right| \leq \sum_{j=2}^n |a_{1,j}||v_{k,j}| \leq M_k \sum_{j=2}^n |a_{1,j}| \\
                |v_{k+1, 1}| &\leq \frac{M_k}{|a_{1,1}|} \sum_{j=2}^n |a_{1,j}| \leq \tilde{\alpha}M_k
            \end{align*}
            $|v_{k+1, j}| \leq \tilde{\alpha} M_k\;(j=1,2,\dots,l)\;(l\in\{1,2,\dots,n-1\})$ ならば $|v_{k+1, l+1}| \leq \tilde{\alpha} M_k$ であることを示す。
            式 (2) の $l+1$ 行目を展開すると次式を得る。
            \begin{align*}
                \sum_{j=1}^{l+1} a_{l+1,j} v_{k+1, j} &= -\sum_{j=l+2}^n a_{l+1,j}v_{k,j} \\
                a_{l+1,l+1} v_{k+1, l+1} &= -\sum_{j=1}^l a_{l+1,j} v_{k+1, j} - \sum_{j=l+2}^n a_{l+1,j}v_{k,j} \\
                |a_{l+1,l+1}| |v_{k+1, l+1}| &= \left|-\sum_{j=1}^l a_{l+1,j} v_{k+1, j} - \sum_{j=l+2}^n a_{l+1,j}v_{k,j}\right| \leq \sum_{j=1}^l |a_{l+1,j}||v_{k+1, j}| + \sum_{j=l+2}^n |a_{l+1,j}||v_{k,j}| \\
                &\leq M_k \sum_{j=1,\dots,n \wedge j\neq l+1} |a_{l+1,j}| \\
                |v_{k+1, l+1}| &\leq \frac{M_k}{|a_{l+1,l+1}|} \sum_{j=1,\dots,n \wedge j\neq l+1} |a_{l+1,j}| \leq \tilde{\alpha} M_k
            \end{align*}
            以上より帰納的に $|v_{k+1,j}| \leq \tilde{\alpha} M_k\;(j=1,2,\dots,n)$ が成り立つ。
            すなわち $M_{k+1} \leq \tilde{\alpha} M_k$ が成り立つ。
            $\tilde{\alpha} < \alpha < 1$ となるように $\alpha $ を定めることで 2. が示される。
        \end{proof}

    \chapter{Cholesky分解}
    \section{rank-one update}
        \begin{itembox}[l]{主張}
            $n\in\naturalNumbers,\;A\in\complexNumbers^{n\times n},\;A\succeq O,\;\bm{x}\in\complexNumbers^n$とし、$A$はHermite行列であるとする。
            $A+\bm{x}\bm{x}^*$に対してCholesky分解のアルゴリズムを適用すると$O(n^3)$の計算量を要する。
            しかし、$A$のCholesky分解$LL^*$が既に得られているとき、$A+\bm{x}\bm{x}^*$のCholesky分解を$O(n^2)$で得ることができる。
            $\bm{x}\bm{x}^*$の階数が1以下である(特に0となるのは$\bm{x}=\bm{0}$の時かつその時に限る)ことから、この方法は ``rank-one update'' と呼ばれている。
        \end{itembox}
        \begin{derivation*}
            方針としては、$n\times n$行列の rank-one update を$(n-1)\times(n-1)$行列の問題に帰着させ、以降同様に逐次的に行列の次数を縮小しながら解を構築する。
            このアルゴリズムの総計算量が$O(n^2)$となるのは明らかであろう。
            \par
            $A+\bm{x}\bm{x}^*$のCholesky分解を$FF^*$とする。
            $L$の第$i$列ベクトルを$\bm{l}_i = [0,\dots,0,l_{i,i},\dots,l_{n,i}]^\top\in\complexNumbers^{n\times n}$とし、同様に$F$の第$i$列ベクトルを$\bm{f}_i = [0,\dots,0,f_{i,i},\dots,f_{n,i}]^\top\in\complexNumbers^{n\times n}$とすると次式が成り立つ。
            \begin{align*}
                \sum_{i=1}^n \bm{f}_i\bm{f}_i^* &= \bm{x}\bm{x}^* + \sum_{i=1}^n \bm{l}_i\bm{l}_i^* \\
                \bm{f}_1\bm{f}_1^* + \sum_{i=2}^n \bm{f}_i\bm{f}_i^* &= \bm{x}\bm{x}^* + \bm{l}_1\bm{l}_1^* + \sum_{i=2}^n \bm{l}_i\bm{l}_i^* \tag{1}
            \end{align*}
            $\bm{f}_i\bm{f}_i^*,\;\bm{l}_i\bm{l}_i^*\;(i=2,3,\dots,n)$の第1行および第1列は0であるから、$\bm{f}_1\bm{f}_1^*$と$\bm{x}\bm{x}^* + \bm{l}_1\bm{l}_1^*$の第1行および第1列が一致する。
            これより次式が成り立つ。
            \[ f_{1,1} = \sqrt{l_{1,1}^2 + |x_1|^2} \eqqcolon r,\; f_{k,1} = \frac{1}{r}\left(l_{1,1}l_{k,1} + \overline{x_1}x_k\right) \; (k=2,3,\dots,n) \tag{2} \]
            ただし$L$の対角成分が非負の実数であることを前提としている。
            以上より、$\tilde{\bm{l}_1} \coloneqq [0,l_{2,1},l_{3,1},\dots,l_{n,1}]^\top,\;\tilde{\bm{x}} \coloneqq [0,x_2,x_3,\dots,x_n]^\top$とすると次式が成り立つ。
            \[ \bm{f}_1 = r\bm{e}_1 + \frac{l_{1,1}}{r}\tilde{\bm{l}_1} + \frac{\conj{x_1}}{r}\tilde{\bm{x}} \]
            ここに$\bm{e}_1$は第1要素が1で他は0であるベクトルである。
            $\bm{f}_1\bm{f}_1^*$の右下$(n-1)\times(n-1)$行列を評価すると次式を得る。
            \begin{align*}
                &\phantom{=} \frac{1}{r^2}\left(l_{1,1}\tilde{\bm{l}_1} + \conj{x_1}\tilde{\bm{x}}\right) = \frac{1}{r^2} \left(l_{1,1}^2\tilde{\bm{l}_1}\tilde{\bm{l}_1}^* + l_{1,1}x_1\tilde{\bm{l}_1}\tilde{\bm{x}}^* + \abs{x_1}^2\tilde{\bm{x}}\tilde{\bm{x}}^* + l_{1,1}\overline{x_1}\tilde{\bm{x}}\tilde{\bm{l}_1}^*\right) \\
                &= \left(1 - \frac{\abs{x_1}^2}{r^2}\right)\tilde{\bm{l}_1}\tilde{\bm{l}_1}^* + \frac{l_{1,1}x_1}{r^2}\tilde{\bm{l}_1}\tilde{\bm{x}}^* + \left(1 - \frac{l_{1,1}^2}{r^2}\right)\tilde{\bm{x}}\tilde{\bm{x}}^* + \frac{\overline{x_1}l_{1,1}}{r^2}\tilde{\bm{x}}\tilde{\bm{l}_1}^* \\
                &= \tilde{\bm{l}_1}\tilde{\bm{l}_1}^* + \tilde{\bm{x}}\tilde{\bm{x}}^* - \frac{1}{r^2}\left(\abs{x_1}^2\tilde{\bm{l}_1}\tilde{\bm{l}_1}^* + l_{1,1}^2\tilde{\bm{x}}\tilde{\bm{x}}^* - x_1l_{1,1}\tilde{\bm{l}_1}\tilde{\bm{x}}^* - \overline{x_1}l_{1,1}\tilde{\bm{x}}\tilde{\bm{l}_1}^*\right) \\
                &= \tilde{\bm{l}_1}\tilde{\bm{l}_1}^* + \tilde{\bm{x}}\tilde{\bm{x}}^* - \bm{y}\bm{y}^* \quad \text{where} \quad \bm{y} = \frac{1}{r}\left(l_{1,1}\tilde{\bm{x}} - x_1\tilde{\bm{l}_1}\right)
            \end{align*}
            上式の$\tilde{\bm{l}_1}\tilde{\bm{l}_1}^* + \tilde{\bm{x}}\tilde{\bm{x}}^*$は$\bm{x}\bm{x}^* + \bm{l}_1\bm{l}_1^*$の右下$(n-1)\times(n-1)$行列である。
            以上より次式が成り立つ。
            \[ \bm{f}_1\bm{f}_1^* = \bm{x}\bm{x}^* + \bm{l}_1\bm{l}_1^* - \bm{y}\bm{y}^* \]
            これを式(1)に適用して次式を得る。
            \[ \sum_{i=2}^n \bm{f}_i\bm{f}_i^* = \bm{y}\bm{y}^* + \sum_{i=2}^n \bm{l}_i\bm{l}_i^* \]
            これは$(n-1)\times(n-1)$行列の rank-one update である。
            このようにして行列の次数を逐次的に縮小し、最後はスカラーの計算に帰着する。
            次数$k$の問題に対し式(2)の計算量は$O(k)$であるから、このアルゴリズムの総計算量は$n(n+1)/2$に比例する。
            \qed
        \end{derivation*}
        このアルゴリズムのJulia 1.8.0による実装例を\inlineCode{Cholesky-decomposition_rank-one_update.ipynb}に記した。
        本文書のGitリポジトリ内でファイル検索すれば見つかる。

    \chapter{LDL分解}
    \section{rank-one update}
        \begin{itembox}[l]{主張}
            $n\in\naturalNumbers,\;A\in\complexNumbers^{n\times n},\;A\succeq O,\;\bm{x}\in\complexNumbers^n$とし、$A$はHermite行列であるとする。
            $A+\bm{x}\bm{x}^*$に対してLDL分解のアルゴリズムを適用すると$O(n^3)$の計算量を要する。
            しかし、$A$のLDL分解$LDL^*$が既に得られているとき、$A+\bm{x}\bm{x}^*$のLDL分解を$O(n^2)$で得ることができる。
            $\bm{x}\bm{x}^*$の階数が1以下である(特に0となるのは$\bm{x}=\bm{0}$の時かつその時に限る)ことから、この方法は ``rank-one update'' と呼ばれている。
        \end{itembox}
        \begin{derivation*}
            方針はCholesky分解の rank-one update と同様である。
            $A+\bm{x}\bm{x}^*$のLDL分解を$FGF^*$とする。
            $D,G$の第$i$対角成分をそれぞれ$d_i,g_i$とする。
            但し$d_i\geq 0$を前提とする。
            $L$の第$i$列ベクトルを$\bm{l}_i = [0,\dots,0,1,l_{i+1,i},\dots,l_{n,i}]^\top\in\complexNumbers^{n\times n}$とし、同様に$F$の第$i$列ベクトルを$\bm{f}_i = [0,\dots,0,1,f_{i+1,i},\dots,f_{n,i}]^\top\in\complexNumbers^{n\times n}$とすると次式が成り立つ。
            \begin{align*}
                \sum_{i=1}^n \bm{f}_i g_i\bm{f}_i^* &= \bm{x}\bm{x}^* + \sum_{i=1}^n \bm{l}_i d_i\bm{l}_i^* \\
                \bm{f}_1 g_1\bm{f}_1^* + \sum_{i=2}^n \bm{f}_i g_i\bm{f}_i^* &= \bm{x}\bm{x}^* + \bm{l}_1 d_1\bm{l}_1^* + \sum_{i=2}^n \bm{l}_i d_i\bm{l}_i^* \tag{1}
            \end{align*}
            $\bm{f}_i g_i\bm{f}_i^*,\;\bm{l}_i d_i\bm{l}_i^*\;(i=2,3,\dots,n)$の第1行および第1列は0であるから、$\bm{f}_1 g_1\bm{f}_1^*$と$\bm{x}\bm{x}^* + \bm{l}_1 d_1\bm{l}_1^*$の第1行および第1列が一致する。
            これより次式が成り立つ。
            \[ g_1 = d_1 + \abs{x_1}^2 \eqqcolon g,\; f_{k,1} = \frac{1}{g}\left(d_1 l_{k,1} + \overline{x_1}x_k\right) \; (k=2,3,\dots,n) \tag{2} \]
            以上より、$\tilde{\bm{l}_1} \coloneqq [0,l_{2,1},l_{3,1},\dots,l_{n,1}]^\top,\;\tilde{\bm{x}} \coloneqq [0,x_2,x_3,\dots,x_n]^\top$とすると次式が成り立つ。
            \[ \bm{f}_1 = \bm{e}_1 + \frac{d_1}{g}\tilde{\bm{l}_1} + \frac{\conj{x_1}}{g}\tilde{\bm{x}} \]
            ここに$\bm{e}_1$は第1要素が1で他は0であるベクトルである。
            $\bm{f}_1 g_1\bm{f}_1^*$の右下$(n-1)\times(n-1)$行列を評価すると次式を得る。
            \begin{align*}
                &\phantom{=} \frac{1}{g}\left(d_1\tilde{\bm{l}_1} + \tilde{\bm{x}}\tilde{\bm{x}}^*\right)\left(d_1\tilde{\bm{l}_1} + \tilde{\bm{x}}\tilde{\bm{x}}^*\right)^* = \frac{d_1}{g}\tilde{\bm{l}_1}d_1\tilde{\bm{l}_1}^* + \frac{\abs{x_1}^2}{g}\tilde{\bm{x}}\tilde{\bm{x}}^* + \frac{d_1}{g}\left(x_1\tilde{\bm{l}_1}\tilde{\bm{x}}^* + \conj{x_1}\tilde{\bm{x}}\tilde{\bm{l}_1}^*\right) \\
                &= \frac{g - \abs{x_1}^2}{g}\tilde{\bm{l}_1}d_1\tilde{\bm{l}_1}^* + \frac{g-d_1}{g}\tilde{\bm{x}}\tilde{\bm{x}}^* + \frac{d_1}{g}\left(x_1\tilde{\bm{l}_1}\tilde{\bm{x}}^* + \conj{x_1}\tilde{\bm{x}}\tilde{\bm{l}_1}^*\right) \\
                &= \tilde{\bm{l}_1}d_1\tilde{\bm{l}_1}^* + \tilde{\bm{x}}\tilde{\bm{x}}^* - \frac{d_1}{g}\left[\abs{x_1}^2\tilde{\bm{l}_1}\tilde{\bm{l}_1}^* + \tilde{\bm{x}}\tilde{\bm{l}_1}^* + x_1\tilde{\bm{l}_1}\tilde{\bm{x}}^* + \conj{x_1}\tilde{\bm{x}}\tilde{\bm{l}_1}^*\right] \\
                &= \tilde{\bm{l}_1}d_1\tilde{\bm{l}_1}^* + \tilde{\bm{x}}\tilde{\bm{x}}^* - \bm{y}\frac{d_1}{g}\bm{y}^* \quad \text{where} \quad \bm{y} = x_1\tilde{\bm{l}_1} + \tilde{\bm{x}}
            \end{align*}
            上式の$\tilde{\bm{l}_1}d_1\tilde{\bm{l}_1}^* + \tilde{\bm{x}}\tilde{\bm{x}}^*$は$\bm{x}\bm{x}^* + \bm{l}_1 d_1\bm{l}_1^*$の右下$(n-1)\times(n-1)$行列である。
            以上より次式が成り立つ。
            \[ \bm{f}_1 g_1\bm{f}_1^* = \bm{x}\bm{x}^* + \bm{l}_1 d_1\bm{l}_1^* - \bm{y}\frac{d_1}{g}\bm{y}^* \]
            これを式(1)に適用して次式を得る。
            \[ \sum_{i=2}^n \bm{f}_i g_i\bm{f}_i^* = \bm{y}\frac{d_1}{g}\bm{y}^* + \sum_{i=2}^n \bm{l}_i d_i\bm{l}_i^* \]
            これは$(n-1)\times(n-1)$行列の rank-one update である。
            このようにして行列の次数を逐次的に縮小し、最後はスカラーの計算に帰着する。
            次数$k$の問題に対し式(2)の計算量は$O(k)$であるから、このアルゴリズムの総計算量は$n(n+1)/2$に比例する。
            \qed
        \end{derivation*}


\end{document}
